One key question that remains is how well provenance captured in a distributed
ledger can be used to quickly create new workflows based on the original.
Duplicating a transaction is a relatively straightforward thing and, in
principle, executing it might be straightforward as well. However, it might be
necessary to investigate how the new transaction can point to the original
transaction from which it was copied. Would it be as simple as adding a
secondary hash in the saved state that pointed to the parent? Or would it
require a ``blocktree'' that allowed branching instead of solely a
blockchain/hash list? In any case, the identity of the parent would seem to be
important provenance itself!

Another important question is how deep the provenance tree could go.
Investigating the degree to which contracts could convert provenance information
to standard ledger state is an interesting topic for further research since it
would address compatibility issues between workflow and provenance models.

Addressing these and other questions as part of an extensive pilot project would
be valuable. There are many open source distributed ledger frameworks, including
the previously mentioned Hyperledger, and a number of commercial solutions that
could keep the investigation focused on the science rather than extensive
software engineering. Success in this endeavor could create an entirely new way
of producing reproducible and reusable scientific workflows.