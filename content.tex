\subsection{Sharing Provenance}
Consider the following statement about the provenance of a piece of data $A$:
\begin{displayquote}
    $A$ was generated yesterday using linear regression on $B$ and stored in
    file $C$.
\end{displayquote}
Provenance capture as it pertains to information, in this case $A$, is the
process of collecting metadata about the creation, manipulation, and use of that
information. This is important when the history of $A$ needs to be trusted and
verified or in situations where $A$ needs to be reused or reproduced reliably.
The provenance in this case describes how $A$ was generated (linear regression),
what data it was generated from ($B$), where it was stored ($C$), the nature of
that storage (a file), and when these things occurred (yesterday).

The previous example could also be taken as a description of a very basic
scientific workflow describing how to manipulate $B$ to generate $A$ and stage
the results in file $C$. The overlap is this close only in the most basic
examples. One key difference between these perspectives is that the workflow
description is executable in some workflow management system, but the provenance
is a persistent record about that workflow. Over time the provenance trail for
$A$ might also encompass workflows across many executions or multiple workflow
management systems and have additional preservation requirements. This
conceptual proximity leads many workflow management systems to include complete
workflow descriptions in provenance records, and some workflow management
systems can recreate and execute workflows from their provenance records alone.

However, one significant problem for provenance as it relates to scientific
workflows is that there are no standard ways to share it. Although some workflow
management systems provide integrated provenance repositories, the practice is
not universal and the provenance repositories might not be designed with other
systems in mind. What if a highly cited publication relied on results generated
from a scientific workflow that could be reproduced quickly if its full history
was available? Furthermore, what if it were possible to compute new and equally
significant results in record time by using the history to generate a new
workflow description with minor modifications to the original? Both of these may
be relatively simple to accomplish in a single, well-designed workflow
management system that captures provenance, but there is no readily available
general-purpose solution.

This work argues that a \textit{permissioned distributed ledger} is an efficient
solution to share workflow provenance between networked peers. The distributed
ledger would act as the broker between peers to execute workflows as
transactions on the network, and each transaction would be recorded in the
ledger as usual. The act of recording the transactions would, in effect, be
capturing the root node of the provenance graph, which would be augmented by
additional information stored in the transaction record to point to
system-specific provenance information. The benefits to this are that
transactions in distributed ledgers are i) shared across the network to
establish consensus on their validity and ii) made immutable by encoding the
state in a distributed hash list, which ensures integrity. Trust among peers is
established by cryptographic keys in permissioned ledgers (as opposed to
proof-of-work schemes), so only minor resources are required to join the
network.

The remainder of this section investigates this possibility. Section
\ref{provenance-background} provides a general overview of provenance, as well
as the basics behind distributed ledgers. Section \ref{case} presents the case for using
distributed ledgers for this purpose in more detail.

\subsection{Workflow Provenance Background}
\label{provenance-background}

The issues around provenance for workflows, including scientific workflows, have
been thoroughly investigated elsewhere in the literature. Data provenance for
workflows has been investigated particularly thoroughly in the data science
community (c.f. \cite{davidson_provenance_2007}), as well as with workflow
management systems for modeling and simulation provenance capture. However,
some systems capture provenance by explicitly generating graph-based
provenance models \cite{pizzi_aiida:_2016}, while others follow a ``log
everything'' style \cite{billings_eclipse_2017}.

Multiple provenance standards exist, most of which are applicable to problems in
broader scientific domains, as well as workflow science. The PROV Model, for
example, is a standard for capturing provenance developed as a successor to the
Open Provenance Model \cite{noauthor_prov-overview_nodate}
\cite{moreau_open_2011}. In PROV, provenance is modeled with a data model,
PROV-DM, and can be stored in XML, RDF, Dublin Core, and a human-readable form
for examples. Constraints can also be applied using the PROV-CONSTRAINTS module.
Provenance is stored in a tree in PROV, and the data model includes entities,
activities, usage, generation, time, and other elements.

In tree-based provenance models such as PROV, provenance is captured in a tree
beginning with an element, usually an entity, at the root node. Entities are
normally the root node because they are often the end product or are used as the
source to generate other data. Depending on the model, the element at the root
node will be either the initial element or the final entity. For example, PROV
roots the provenance tree in the final created entity, and edges in the tree
indicate usage or generation going backwards in time from the end to the
beginning. This is a natural representation when considering the question
``Where did this element come from?'', whereas a time-forward perspective is
natural when considering the question ``What did this element become or
produce?'' Activities in tree-based provenance models are models as nodes, like
entities.

Event-based or ``log everything'' provenance models provide detailed provenance
information based on logs created from events in the system. Workflow management
systems can easily generate this type of provenance trail as the workflow is
executed (which is why it was chosen in \cite{billings_eclipse_2017}). As the
workflow is executed, the input and output are captured at each step, as well as
a description of or the entire instruction set, and logged. The entire workflow
description might be saved as well. Information on activities, entities, time,
and other elements can be captured the same as a tree-based provenance model.
The provenance information is most commonly available in the form of logs but
might be generated as separate provenance reports in some systems. Although it
is conceivable that a provenance trail in an event-based system could be ordered
with time running backwards, as with all event-based systems it is much more
common to find a time-forward representation. Some systems, including those in
\cite{billings_eclipse_2017} and \cite{altintas_provenance_2006}, can use this
type of provenance record to enable ``fast replays'' of workflows.

\subsubsection{Distributed Ledgers}
Distributed ledgers are linked collections of records about transactions that
are distributed across a peer-to-peer network without any central authority. The
Blockchain data structure that forms the basis of the BitCoin cryptocurrency is
the most well-known implementation of a distributed ledger
\cite{nakamoto_bitcoin:_nodate}. Records, or groups of records called Blocks,
are linked in order through a hash list where each item is linked to the one
before and after it through a unique hash of the record(s), forming a ``chain.''
Without a central authority to certify the validity of records, the means of
determining consensus on whether or not records are valid requires the use of a
consensus algorithm executed by nodes in the network. The exact algorithm used
depends on whether or not the ledger is open or permissioned. Open networks,
such as cryptocurrencies, tend to determine consensus through proof-of-work
algorithms \cite{nakamoto_bitcoin:_nodate} or through proof of stake
\cite{noauthor_proof--stake_2018}.

The basic operation of a distributed ledger is as follows, assuming an
underlying Blockchain implementation:
\begin{itemize}
    \item A transaction is executed on the network between entities $A$ and $B$
    to create asset $C$.
    \item The transaction is time-stamped and added to a collection of
    unverified transactions, which are linked to previously verified
    transactions through a hash.
    \item A check (such as proof of work) is executed on the collected
    transactions to ascertain their validity.
    \item The majority of nodes in the remainder of the network comes to a
    consensus on the result of the check as presented by the original node or
    nodes that checked the collection.
    \item The collection is accepted, and its hash is used for the next
    collection; the process is then repeated.
\end{itemize}

The most appealing features of distributed ledgers are that a central authority
is not required and that the combination of consensus algorithms and a hash list
(or hash tree) to verify and store the transactions creates a very reliable
system. Because of this, distributed ledgers have found application in
cryptocurrencies, traditional financial markets, and many other areas
\cite{noauthor_groups:requirements:use-case-inventory_nodate}
\cite{noauthor_discussion_2018}. The applicability of Blockchain to business
process modeling has also been investigated
\cite{mendling_blockchains_2018}.

\subsubsection{Permissioned Versus Open Networks}
Open networks must use proof-based algorithms to establish trust because members
of the network are inherently untrustworthy. Thus, by providing proof that is
acceptable to a majority of the remaining network, nodes can be added to the
list. Since the proof might be computationally expensive, cryptocurrencies
provide the incentive of receiving coins in the currency appropriate to the
work.

If the primary motivation of such a costly proof scheme is to create trust
between untrustworthy parties, one obvious alternative is to work only with
trustworthy nodes. Permissioned networks are formed by nodes identified by
strong cryptographic keys and allowed to join by permission of other members in
the network. Reaching consensus in this situation is as simple as making sure
that the source of the transaction and its purpose are valid between some
parties on the network. No incentive is required to check transactions and vouch
for them in this case beyond membership in the network, which improves
performance and alleviates any concern about the work, stake, or cost required
to participate.

\subsubsection{Smart Contracts}
A smart contract, or simply a contract, in this context is a small piece of code
that is executed in response to a transaction. Business logic executed on the
network is done so through contracts. Contracts can have many uses, although
some ledgers might limit the type of code that can be executed for either
architectural reasons or security.

\subsubsection{Relationship to Other Technologies}
A number of technologies are closely related to distributed ledgers.
Blockchain data structures are implemented using Merkle trees
\cite{merkle_digital_1987}, as are other distributed databases and version
control systems, such as Git. Substantial work is on-going to improve
on the performance of distributed ledgers and Blockchains in particular. For
example, the PHANTOM protocol, based on BlockDAG, alleviates many of the
performance issues that are side effects of Nakamoto's original consensus
scheme and allows for asynchronous, fast block creation. Other efforts are also
investigating the use of Blockchain technologies for provenance capture
\cite{richard_brooks_and_anthony_skjellum_using_2017}\cite{worley_2018}.

\subsection{Distributed Ledgers for Workflow Provenance}
\label{case}
The promise of a permissioned ledger for managing workflow provenance is that a
large, secure network of peers can quickly and automatically share provenance
information without additional work and in a way that preserves the integrity
of the provenance record. Although there might be other means to accomplish this
task, permissioned distributed ledgers have additional properties that map well
onto those required of a good provenance record.
\begin{itemize}
    \item Transaction records, which would describe workflow executions in this
    case, are immutable. Placement in the ledger requires linking to the
    previous and next records in the Merkle tree/hash list/Blockchain backing
    the ledger. Thus, the record cannot be changed without changing the entire
    set of records.
    \item Transaction records are secure. In addition to inheriting
    immutability, which is itself a good component of security, records are
    deemed valid by trusted network peers. This offers both security through
    consensus and security through the ``reputation'' of the peers.
    \item Transaction records can hold significant amounts of state. Records
    include information about themselves, such as when they were created, what
    the transaction was meant to do, etc. However, contracts can be used to
    inject additional state into the records, which could include root node
    provenance information, or to provide extra provenance parameters to the
    workflow management system on the supply side of the transaction.
    \item Transaction records are \textit{uniquely} identifiable in the ledger
    and can be found through queries.
    \item Transaction records are \textit{uniformly} identifiable in the ledger,
    meaning that the same method to identify one transaction can be used to
    uniquely identify others. Thus, at a very high level, such a scheme would
    imply that the root of any provenance record could be found without
    requiring knowledge of the exact provenance standard used by the workflow
    management systems at the nodes.
    \item The ledger can be walked easily in either time-backward or
    time-forward order, matching either tree-based or event-based provenance
    models.
\end{itemize}

These properties also suggest that ledgers would be good for managing provenance
regardless of whether or not the ledger was distributed. (Trusted peers could be
queried secretly and their consensus computed locally, for example.)

\subsubsection{Capture Model}
In a distributed ledger, workflow management systems would be peers on the
network, along with clients that represent humans and possibly other service
nodes, such as nodes for staging data. The latter point makes sense because
moving data can itself be considered a part of setup or postprocessing
workflows. Transactions---workflow executions---would occur when one peer in the
network asks another to execute a workflow, which would start by executing a
contract to initiate the actual workflow execution. Once the task was complete,
the transaction would be tested for validity and consensus gained on the
network, at which point details about the workflow execution would be logged.
Workflow management systems could either export the entire provenance record
into the contract code that executed the workflow, which could write it as state
in the network, or add an entry to the state that describes the provenance
standard used by the system and a second entry pointing to the file/resource
that contains the rest of the provenance record. The first case has the benefit
that the provenance record will never be lost as long as the ledger exists, but
the second case might be more scalable because it requires less compute cycles
and stores less data. In practice, these options could be used together without
an obvious downside.

One interesting case to consider is if the client systems are using the workflow
ontology for capturing instance metadata and a provenance model such as PROV.
PROV is RDF-based, just like the workflow ontology, and would merge naturally
into the instance graph for any workflow described by the ontology because of
the Anyone can say Anything about Anything principle. This implies that
providing the whole graph of transactions or providing the information task by
task (transaction by transaction) are equivalent.

\subsubsection{Access}
There is little reason to consider the use of a distributed ledger for workflow
provenance that is not permissioned since presumably all of the peers using the
network would be at scientific institutions of one form or another. Thus, the
numerous issues associated with proof schemes, be they proof of work, stake, or
time, could be avoided by registration alone. The full transaction log could be
shared publicly as well so that others could benefit from the work, even if they
do not contribute to it. However, the question of proprietary data in the ledger
requires a slightly different solution.

Proprietary data or data that cannot otherwise be revealed to the public can be
stored in a distributed ledger and is not necessarily a reason to choose either
an open or permissioned ledger for scientific workflows. Hyperledger, for
example, solves the issue of proprietary network in the ledger by introducing
\textit{channels} \cite{noauthor_hyperledger_nodate}. Private transactions are
handled in private channels, and although the existence of the transaction is
recorded in the ledger, no state is saved for that transaction, which hides all
the secret information. This is a nice alternative that makes it possible to
share some information while saving other information instead of completely
hidding the ledger behind permissioning or adopting another technology.